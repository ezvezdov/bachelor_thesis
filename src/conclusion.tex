%!TEX root = ../main.tex

\chapter{Conclusion DRAFT\label{chap:conclusion}}

% In this thesis we focused on comparing traditional methods for text representation and transformer-based methods.
% We reviewed architectures, work principles, strengths, limitation, uniqueness and purposes of the traditional methods of text representations.
% We reviewed Transformer architecture.
% We describe BERT model and explain its training process. Also we described BERT strenghts.
% We reviewed two evaluation methods to measure quality of the models for our task.
% After we reviewed ideas of \ac{RAG} and principles of work. After we reviewed \ac{RAG} architecture and discuss about parameters, that can affect accuracy and rightness of its work.

% Next we discussed about chosen language and its grammatical representation, in result we choosed diacritics and diacriticless versions of texts.
% Next was review of UPV FAQ benchmark.
% After we defined baseline using fasttext model.
% Next we chose transformer based models to evaluate.
% In result we choosed 15 groups of models that expands to 37 models.

% After we discussed parameters and evaluation process of \ac{RAG} and decided to test 5 sizes of chunks of to use in \ac{RAG}.

% Next we present evaluation results, where most models have above deadline results.
% We discuss about models performances and reasons its performances.
% We define best model of the evaluation that was \ac{mE5}\textsubscript{Large} and present confussion matrix of its evaluation of one of the benchmark subset.
% After result analys we defined balanced models, that has best performances for its model size.
% That models are SimCSE-RetroMAE-Small, small version of \ac{GTE} and all sized of \ac{mE5}.

% In second part of evaluation we focused on evaluation \ac{RAG} algorithm with different chunk size, wherein computing its factuality and computational efficiency.
% We define, that the best chunk size is TODO.
% We analyze this evaluation TODO.

This thesis investigated the effectiveness of transformer-based models for text representation compared to traditional methods in the context of semantic similarity assessment for Czech text.

The analysis began with a comprehensive review of traditional text representation methods.
Their architectures, underlying principles, strengths, weaknesses, and specific applications were examined.
This review established a foundation for understanding the evolution of text representation techniques.

Following this, the study shifted its focus to transformer architectures.
Here, the investigation delved into the inner workings of these models and explored their advantages over traditional methods.
The \ac{BERT} model served as a specific example, with an explanation of its training process and its strengths in capturing semantic meaning from textual data.

To ensure objective assessment of model quality for the chosen task, two relevant evaluation methods were reviewed.
These established methods provided a framework for comparing the performance of different text representation models used for semantic similarity assessment in Czech text.

Next, the investigation explored the \ac{RAG} model.
Core concepts, operational principles, and critical parameters influencing \ac{RAG}'s accuracy were examined.
This in-depth analysis proved crucial for effectively configuring and evaluating \ac{RAG} within the context of the chosen task.

Recognizing the importance of language-specific analysis, Czech was chosen as the target language.
The study incorporated both diacritic and diacritic-less text versions to account for potential variations within Czech text data.
The established UPV FAQ benchmark served as the standard for consistent and reliable evaluation.

A baseline performance metric was established using the fastText model.
This baseline provided a benchmark for comparing the performance of the transformer-based models.
Following the establishment of this baseline, a diverse selection of 15 transformer-based model groups (encompassing a total of 37 models) were chosen for further evaluation.

The \ac{RAG} evaluation process involved testing five different chunk sizes.
This exploration aimed to understand the impact of chunk size on both the factuality (accuracy and relevance) of retrieved information and computational efficiency (processing time).
While the optimal chunk size is still under investigation, further analysis is ongoing.
These findings will be instrumental in refining \ac{RAG}'s configuration for future applications.

The evaluation results revealed that a significant portion of the transformer-based models outperformed the baseline, suggesting their promise for semantic similarity assessment in Czech text.
A detailed analysis of model performance and influencing factors identified \ac{mE5} \textsubscript{Large} as the top performer.
A confusion matrix visualized its evaluation on a specific benchmark subset.
Additionally, "balanced models" exhibiting the best performance relative to their model size were highlighted.
These included SimCSE-RetroMAE-Small, the small version of \ac{GTE}, and all sizes of \ac{mE5}, demonstrating the potential of both large and efficient models for this task.


TODO: ADD RAG
% In second part of evaluation we focused on evaluation \ac{RAG} algorithm with different chunk size, wherein computing its factuality and computational efficiency.
% We define, that the best chunk size is TODO.
% We analyze this evaluation TODO.




% \section{TODO: REMOVE CONCLUSION STRUCTURE}
% \begin{itemize}
%     \item Summarize the key takeaways from the research, emphasizing the most effective text representation model for RAG in technical QA based on the evaluation criteria.
%     \item Briefly mention the trade-offs between factuality, CPU usage, and other factors in selecting representations for RAG.
%     \item Suggest potential future research directions, such as exploring other text representation methods or evaluating RAG performance on different datasets.
% \end{itemize}
