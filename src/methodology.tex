%!TEX root = ../main.tex

\chapter{Methodology\label{chap:methodology}}

\section{Evaluation process}

This work specifically targets the evaluation of word, sentence, and paragraph representation methods on datasets in the Czech language.
This focus on Czech allows for a deeper understanding of how these methods perform in a language with specific characteristics, such as a rich inflectional morphology and the presence of diacritics.

\subsection{Text Data Preparation}
In certain foreign languages, a common issue arises when individuals incorrectly write words by omitting diacritics or altering letters, \reffig{fig:diacritics_diacriticless}.
This problem is prevalent in social media, chatbots, and other informal written communications.
As a result, embedding models face challenges in comprehending text without diacritics (hereinafter diacriticless). A potential solution involves adapting data representation to accommodate both formal and informal styles of writing.

\begin{figure}[h]
  \centering
  \begin{tikzpicture}
    %Nodes
    \begin{scope}[every node/.style={rectangle, draw=black, thick, align=left, minimum width=12em, minimum height=1.7em}]
    \node (acute_diacritics)       []                             {á $\vert$ é $\vert$ í $\vert$ ó $\vert$ ú $\vert$ ý};
    \node (acute_diacriticless)    [right=of acute_diacritics]    {a $\vert$ e $\vert$ i $\vert$ o $\vert$ u $\vert$ y};
    \node (caron_diacritics)       [below=of acute_diacritics]    {č $\vert$ ď $\vert$ ě $\vert$ ň $\vert$ ř $\vert$ š $\vert$ ť $\vert$ ž};
    \node (caron_diacriticless)    [right=of caron_diacritics]    {c $\vert$ d $\vert$ e $\vert$ n $\vert$ r $\vert$ s $\vert$ t $\vert$ z};
    \node (overring_diacritics)    [below=of caron_diacritics]    {ů};
    \node (overring_diacriticless) [right=of overring_diacritics] {u};
    \end{scope}

    % Text labels (outside nodes)
    \node[left=of acute_diacritics.west]    {Acute:};
    \node[left=of caron_diacritics.west]    {Caron:};
    \node[left=of overring_diacritics.west] {Overring:};

    %Lines
    \draw[->] (acute_diacritics.east)    -- (acute_diacriticless.west);
    \draw[->] (caron_diacritics.east)    -- (caron_diacriticless.west);
    \draw[->] (overring_diacritics.east) -- (overring_diacriticless.west);
\end{tikzpicture}
  \caption{Usual changes in informal czech texts.}
  \label{fig:diacritics_diacriticless}
\end{figure} 

This study will employ two distinct text representations: text with diacritics and text without diacritics.
To ensure optimal evaluation, the diacritic text will be assessed using datasets that preserve these diacritics, while the diacriticless text will be evaluated using datasets that lack diacritics.

As detailed in \reflst{lst:diacriticless.sh}, this script is used for creating diacriticless versions of the datasets.

\begin{lstlisting}[language=bash,basicstyle=\small\ttfamily, frame=single, caption={Script for removing diacritics using Unix utilities}, captionpos=b, label={lst:diacriticless.sh},backgroundcolor=\color{light-gray}] 
  sed 's/.*/\L&/' "$1" | iconv -f utf-8 -t ascii//TRANSLIT > diacriticless/"$1"
\end{lstlisting}

\subsection{Datasets}

\subsubsection{UPV FAQ}

Dataset is comprised of frequently asked questions (FAQs) and their answers from the Industrial Property Office of the Czech Republic website, organized into four distinct groups, as detailed in \reftab{table:UPV_FAQ_info}.

This work will evaluate two key metrics using UPV FAQ dataset
\begin{itemize}
  \item \textbf{Question matching accuracy}: This metric involves calculating the cosine similarity between all possible question pairs within a dataset.
                                            A question is considered successfully matched if its second-highest cosine similarity score corresponds to another question belonging to the same class (i.e., the question with the highest similarity is likely the same question itself). 
                                            The overall question matching accuracy is then computed as the ratio of successfully matched questions to the total number of question pairs evaluated.  
  \item \textbf{Answer matching accuracy}: This metric assesses the system's ability to identify the correct answer for a given question.
                                            The system accomplishes this by directly comparing the question with pre-generated answer embeddings.
                                            By evaluating the similarity between the question and each answer embedding, the system classifies the question as corresponding to the answer with the highest similarity score.
                                            The overall answer matching accuracy is then calculated as the proportion of questions for which the system correctly identifies the corresponding answer.  
\end{itemize}

\begin{table}[!ht]
  \centering
  \begin{tabular}{ |p{2cm}||p{3cm}| }
    \hline
    Tests set & Number of tests \\
    \hline
    FAQv5     & 2054            \\
    FAQ50     & 561             \\
    FAQ76     & 2025            \\
    FAQ76v2   & 1965            \\
    \hline
    \hline
    All       & 6605            \\
    \hline
  \end{tabular}
  \caption{Information about UPV FAQ tests} \label{table:UPV_FAQ_info}
\end{table}


\subsection{Baseline}
Due to the inherent morphological richness of the Czech language, this study adopts \textbf{FastText} as the baseline word embedding method.
This decision is motivated by FastText's ability to effectively capture morphological variations within words, a characteristic that has been shown to be advantageous for languages like Czech.
While other techniques like Word2Vec and \ac{GloVe} have been explored for word embedding generation, they have demonstrated lower performance in this context.

\subsubsection{Training parameters}
\begin{itemize}
  \item Architecture: \ac{CBOW}
  \item Vector size: 300
  \item Loss function: Negative sampling loss
  \item Dictionary threshold (the frequency of the word to be included in the dictionary): 130
\end{itemize}

\subsubsection{Training data}

FastText model is trained on a segment of a preprocessed Common Crawl repository, which encompasses raw data from web pages, as well as metadata and text extractions.
Given the nature of this dataset, it is expected to include misspellings and text lacking diacritics.

\begin{enumerate}
  \item Eliminating duplicate entries
  \item Filtering out lines containing fewer than 9 characters
  \item Excluding URLs
  \item Breaking lines into individual sentences
  \item Converting all text to lowercase
  \item Removing lines containing words exceeding 30 characters
  \item Excluding HTML tags with less than 100 characters
  \item Removing lines surpassing 500 characters
  \item Omitting words longer than 21 characters

\end{enumerate}


\subsection{Chosen transformer models}

The evaluation process will leverage a curated selection of embedding models. This selection encompasses two categories:

\begin{enumerate}
  \item \textbf{Existing Czech Embedding Models}: To assess the performance of established solutions within the Czech \ac{NLP} community, we will incorporate existing Czech embedding models into the evaluation.
  \item \textbf{Models from \ac{MTEB}}: We will also evaluate the effectiveness of all multilingual and highly regarded, open-source, English models available through the \ac{MTEB}. This inclusion allows for a comparative analysis of how these models generalize to the Czech language.
\end{enumerate}





\subsubsection{mBERT}
\ac{mBERT} is a \ac{BERT} model, trained on a Wikipedia dump of 100 languages.
The model performs best on high-resource languages such as English, French and Chinese, since lower-resource languages are underrepresented in the training data.
We test whether M-BERT's pre-training on a wide range of languages, and thus a wide range of culture-specific analogies, might enhance the model's general analogy understanding.

\subsubsection{CZERT}

\subsubsection{multilingual-e5}

\subsubsection{UAE-Large-V1}

\subsubsection{distiluse-base-multilingual-cased-v2}

\subsubsection{paraphrase-multilingual-MiniLM-L12-v2}

\subsubsection{paraphrase-multilingual-mpnet-base-v2}

\subsubsection{bge}

\subsubsection{gte}

\subsubsection{LaBSE}

\subsubsection{xlm-roberta}

\subsubsection{snowflake-arctic-embed}

\subsubsection{nomic-embed-text-v1.5}

\subsubsection{Seznam's models}





\begin{itemize}
    \item Describe the evaluation process for different text representations.
    \item \begin{itemize}
        \item{Specify the chosen word, sentence, and paragraph representation models (e.g., FastText, BERT variants).}
        \item{Explain the usage of analogy tests and confusion matrices for evaluation.}
        \item{Detail the selection process for the UPV corpus set and its suitability for technical QA tasks.}
      \end{itemize}
    \item Outline the second part of the study focusing on RAG for technical QA.
    \item \begin{itemize}
        \item{Explain the RAG algorithm and its reliance on text representations.}
        \item{Describe the evaluation approach for selecting optimal representations for RAG.}
        \item{Mention the factors considered during evaluation, such as embedding efficiency, text chunk size, and factuality of generated answers.}
      \end{itemize}
\end{itemize}









