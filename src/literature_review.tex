%!TEX root = ../main.tex

\chapter{Literature Review\label{chap:literature_review}}

There are several methods for word representation.
% These methods can be splitted into traditional non-contextualized methods and 

\section{Traditional word embedding methods}

\subsection{Word2Vec \cite{word2vec}}  
Word2Vec is algorithm that generates word embedding using information about target word (context).
Word2Vec uses \ac{NN} and \ac{ML} techniques to generate word embedding for every word in vocabulary during training.
As \ac{NN} architecture are used \ac{CBOW} and Skip-gram, \reffig{fig:cbow_skipgram_scheme}.

\begin{figure}[h]
    \centering
    \begin{tikzpicture}[
    framed,
    squarednode/.style={rectangle, draw=black!60, very thick, minimum size=5mm},
    emptysquare/.style={squarednode, draw=none},
    largedrectangle/.style={squarednode, minimum width=10mm, minimum height=15mm},
    emptyrectangle/.style={squarednode, draw=none,minimum width=10mm},
    ]
    %Nodes
    \node[emptysquare]     (input3)  []                 {};
    \node[squarednode]     (input2)  [above=of input3]  {};
    \node[squarednode]     (input1)  [above=of input2]  {};
    \node[squarednode]     (input4)  [below=of input3]  {};
    \node[squarednode]     (input5)  [below=of input4]  {};
    \node[largedrectangle] (sum)     [right=of input3]  {SUM};
    \node[emptyrectangle]  (sum3)    [right=of input3]  {};
    \node[emptyrectangle]  (sum2)    [above=of sum3]    {};
    \node[emptyrectangle]  (sum1)    [above=of sum2]    {};
    \node[squarednode]     (output)  [right=of sum]     {};
    \node[emptysquare]     (output2) [above=of output]  {};
    \node[emptysquare]     (output1) [above=of output2] {};
    
    % Text labels (outside nodes)
    \node[left=of input1.east, xshift=3mm] {w(t-2)};
    \node[left=of input2.east, xshift=3mm] {w(t-1)};
    \node[left=of input4.east, xshift=3mm] {w(t+1)};
    \node[left=of input5.east, xshift=3mm] {w(t+2)};
    \node[below=of output.north]           {w(t)};
    \node[above=of input1.north]           {Input};
    \node[above=of sum1.north]             {Projection};
    \node[above=of output1.north]          {Output};

    %Lines
    \draw[->] (input1.east) -- (sum.west);
    \draw[->] (input2.east) -- (sum.west);
    \draw[->] (input4.east) -- (sum.west);
    \draw[->] (input5.east) -- (sum.west);
    \draw[->] (sum.east)    -- (output.west);
\end{tikzpicture}
    \begin{tikzpicture}[
    framed,
    squarednode/.style={rectangle, draw=black!60, very thick, minimum size=5mm},
    emptysquare/.style={squarednode, draw=none},
    largedrectangle/.style={squarednode, minimum width=10mm, minimum height=15mm},
    emptyrectangle/.style={squarednode, draw=none,minimum width=10mm},
    ]
    %Nodes
    \node[squarednode]     (input3)  []                 {};
    \node[emptysquare]     (input2)  [above=of input3]  {};
    \node[emptysquare]     (input1)  [above=of input2]  {};
    \node[emptysquare]     (input4)  [below=of input3]  {};
    \node[emptysquare]     (input5)  [below=of input4]  {};
    \node[largedrectangle] (sum)     [right=of input3]  {};
    \node[emptyrectangle]  (sum3)    [right=of input3]  {};
    \node[emptyrectangle]  (sum2)    [above=of sum3]    {};
    \node[emptyrectangle]  (sum1)    [above=of sum2]    {};
    \node[emptysquare]     (output3) [right=of sum]     {};
    \node[squarednode]     (output2) [above=of output3] {};
    \node[squarednode]     (output1) [above=of output2] {};
    \node[squarednode]     (output4) [below=of output3] {};
    \node[squarednode]     (output5) [below=of output4] {};
    \node[emptysquare]     (output1) [above=of output2] {};
    
    % Text labels (outside nodes)
    \node[below=of input3.north]  {w(t)};
    \node[right=of output1.west]  {w(t-2)};
    \node[right=of output2.west]  {w(t-1)};
    \node[right=of output4.west]  {w(t+1)};
    \node[right=of output5.west]  {w(t+2)};
    \node[above=of input1.north]  {Input};
    \node[above=of sum1.north]    {Projection};
    \node[above=of output1.north] {Output};

    %Lines
    \draw[->] (input3.east) -- (sum.west);
    \draw[->] (sum.east)    -- (output1.west);
    \draw[->] (sum.east)    -- (output2.west);
    \draw[->] (sum.east)    -- (output4.west);
    \draw[->] (sum.east)    -- (output5.west);
\end{tikzpicture}
    \caption{\ac{CBOW} and Skip-gram schemes respectively}
    \label{fig:cbow_skipgram_scheme}
\end{figure} 
  

\subsection{GloVe \cite{glove}}
\subsection{FastText \cite{fasttext}}





\begin{itemize}
    \item Discuss traditional word embedding methods like FastText and their limitations.
    \item Explain the concept of transformer-based models like BERT and their advantages for text representation.
    \item Review related work on RAG algorithms and their dependence on effective text representations.Discuss existing research on evaluating text representations using analogy tests and confusion matrices.
    \item Briefly mention the UPV corpus set as the chosen evaluation benchmark.
\end{itemize}