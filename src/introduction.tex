%!TEX root = ../main.tex

\chapter{Introduction\label{chap:introduction}}

% First, introduce the reader to the research topic.
% Start with the most general view and slowly converge to the particular field, sub-field, and the challenges you face.
% You can cite others' work here \cite{baca2021mrs}.

% \section{Related works}

% This section should contain related state-of-the-art works and their relation to the author's work.
% We usually cite the original works like this \cite{benallegue2008high}.
% You can also cite multiple papers at once like this \cite{baca2016embedded, baca2021mrs}.

% \section{Contributions}

% This section should describe the author's contributions to the field of research.

% \section{Mathematical notation}

% It is a good practice to define basic mathematical notation in the introduction.
% See \reftab{tab:mathematical_notation} for an example.

% \begin{table*}[!h]
%   \scriptsize
%   \centering
%   \noindent\rule{\textwidth}{0.5pt}
%   \begin{tabular}{lll}
%     $\mathbf{x}$, $\bm{\alpha}$ & vector, pseudo-vector, or tuple\\
%     $\mathbf{\hat{x}}$, $\bm{\hat{\omega}}$& unit vector or unit pseudo-vector\\
%     $\mathbf{\hat{e}}_1, \mathbf{\hat{e}}_2, \mathbf{\hat{e}}_3$ & elements of the \emph{standard basis} \\
%     $\mathbf{X}, \bm{\Omega}$ & matrix \\
%     $\mathbf{I}$ & identity matrix \\
%     $x = \mathbf{a}^\intercal\mathbf{b}$ & inner product of $\mathbf{a}$, $\mathbf{b}$ $\in \mathbb{R}^3$\\
%     $\mathbf{x} = \mathbf{a}\times\mathbf{b}$ & cross product of $\mathbf{a}$, $\mathbf{b}$ $\in \mathbb{R}^3$\\
%     $\mathbf{x} = \mathbf{a}\circ\mathbf{b}$ & element-wise product of $\mathbf{a}$, $\mathbf{b}$ $\in \mathbb{R}^3$ \\
%     $\mathbf{x}_{(n)}$ = $\mathbf{x}^\intercal\mathbf{\hat{e}}_n$ & $\mathrm{n}^{\mathrm{th}}$ vector element (row), $\mathbf{x}, \mathbf{e} \in \mathbb{R}^3$\\
%     $\mathbf{X}_{(a,b)}$ & matrix element, (row, column)\\
%     $x_{d}$ & $x_d$ is \emph{desired}, a reference\\
%     $\dot{x}, \ddot{x}, \dot{\ddot{x}}$, $\ddot{\ddot{x}}$ & ${1^{\mathrm{st}}}$, ${2^{\mathrm{nd}}}$, ${3^{\mathrm{rd}}}$, and ${4^{\mathrm{th}}}$ time derivative of $x$\\
%     $x_{[n]}$ & $x$ at the sample $n$ \\
%     $\mathbf{A}, \mathbf{B}, \mathbf{x}$ & LTI system matrix, input matrix and input vector\\
%     \emph{SO(3)} & 3D special orthogonal group of rotations\\
%     \emph{SE(3)} & \emph{SO(3)}~$\times~\mathbb{R}^3$, special Euclidean group\\
%   \end{tabular}
%   \noindent\rule{\textwidth}{0.5pt}
%   \caption{Mathematical notation, nomenclature and notable symbols.}
%   \label{tab:mathematical_notation}
% \end{table*}

\section{Text representation}
The human language, with its nuances and complexities, presents a significant challenge for machines to understand.
\ac{NLP} bridges this gap, and at its core lies the critical concept of text representation.
This process acts as a translator, bridging the gap between the richness of text and the numerical language that machines understand.
By effectively capturing the meaning within words and their relationships, text representation empowers \ac{NLP} models to leverage machine learning's capabilities.
From sentiment analysis to machine translation, this ability to represent meaning fuels the advancements in \ac{NLP}, enabling machines to interact with and decipher human language with ever-increasing accuracy.

\section{Evolution of text representation methods}

\ac{NLP} has undergone a significant transformation in its approach to text representation.
Early methods, such as one-hot encoding, while simple to implement, suffered from limitations in efficiency due to dimensionality and sparsity issues.

Word embedding techniques (e.g., Word2Vec, \ac{GloVe}, FastText) offered a significant improvement by capturing semantic relationships between words through high-dimensional word vectors.
However, these techniques primarily focused on local context within a limited window, hindering their ability to capture complex relationships within sentences or documents.

The emergence of deep learning architectures, particularly transformer-based models like \ac{BERT}, revolutionized the field of text representation.
These models allows to not only understand the meaning of individual words but also consider their interaction and context within a sentence or document.

\begin{figure}
  \centering
  \begin{tikzpicture}
    %Nodes
    \begin{scope}[every node/.style={rectangle, draw=black, thick, align=left, minimum width=10em}]
    \node (Word2Vec)     []                  {Word2Vec};
    \node (GloVe)        [below=of Word2Vec] {\ac{GloVe}};
    \node (FastText)     [below=of GloVe]    {FastText};
    \node (Transformers) [below=of FastText] {Transformers};
    \end{scope}

    % Text labels (outside nodes)
    \node[right=of Word2Vec.east]     {2013};
    \node[right=of GloVe.east]        {2014};
    \node[right=of FastText.east]     {2016};
    \node[right=of Transformers.east] {2017};

    %Lines
    \draw[->] (Word2Vec.south) -- (GloVe.north);
    \draw[->] (GloVe.south)    -- (FastText.north);
    \draw[->] (FastText.south) -- (Transformers.north);
\end{tikzpicture}  
  \caption{Evolution of the text representation methods.}
  \label{fig:ecolution_text_representation}
\end{figure}

\section{Research objective}

This research aims to evaluate the effectiveness of various word, sentence, and paragraph representations for their subsequent application in \ac{RAG} algorithms, with a specific focus on the domain of technical \ac{QA}.