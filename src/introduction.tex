%!TEX root = ../main.tex

\chapter{Introduction\label{chap:introduction}}

\section{Text representation}
The human language, with its nuances and complexities, presents a significant challenge for machines to understand.
\ac{NLP} bridges this gap, and at its core lies the critical concept of text representation.
This process acts as a translator, bridging the gap between the richness of text and the numerical language that machines understand.
By effectively capturing the meaning within words and their relationships, text representation empowers \ac{NLP} models to leverage the capabilities of \ac{ML}.
From sentiment analysis to machine translation, this ability to represent meaning fuels the advancements in \ac{NLP}, enabling machines to interact with and decipher human language with ever-increasing accuracy.

\section{Evolution of text representation methods}

\ac{NLP} has undergone a significant transformation in its approach to text representation.
Early methods, such as one-hot encoding (e.g. \ac{BoW} \cite{wiki_BoW}, \ac{TF-IDF} \cite{wiki_tf_idf}), while simple to implement, suffered from limitations in efficiency due to dimensionality and sparsity issues.

Word embedding techniques (e.g., Word2Vec, \ac{GloVe}, FastText) offered a significant improvement by capturing semantic relationships between words through high-dimensional word vectors.
However, these techniques primarily focused on local context within a limited window, hindering their ability to capture complex relationships within sentences or documents.

The emergence of deep learning architectures, particularly transformer-based models like \ac{BERT}, revolutionized the field of text representation.
These models allow us to not only understand the meaning of individual words but also consider their interaction and context within a sentence or document.

\begin{figure}
 \centering
  \begin{tikzpicture}
    %Nodes
    \begin{scope}[every node/.style={rectangle, draw=black, thick, align=left, minimum width=10em}]
    \node (Word2Vec)     []                  {Word2Vec};
    \node (GloVe)        [below=of Word2Vec] {\ac{GloVe}};
    \node (FastText)     [below=of GloVe]    {FastText};
    \node (Transformers) [below=of FastText] {Transformers};
    \end{scope}

    % Text labels (outside nodes)
    \node[right=of Word2Vec.east]     {2013};
    \node[right=of GloVe.east]        {2014};
    \node[right=of FastText.east]     {2016};
    \node[right=of Transformers.east] {2017};

    %Lines
    \draw[->] (Word2Vec.south) -- (GloVe.north);
    \draw[->] (GloVe.south)    -- (FastText.north);
    \draw[->] (FastText.south) -- (Transformers.north);
\end{tikzpicture}  
 \caption{Evolution of the text representation methods.}
  \label{fig:ecolution_text_representation}
\end{figure}

\section{Research objective}

This research aims to evaluate the effectiveness of various word, sentence, and paragraph representations for their subsequent application in \ac{RAG} algorithms, with a specific focus on the domain of technical \ac{QA}.