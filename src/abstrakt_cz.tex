%!TEX root = ../main.tex

\begin{changemargin}{0.8cm}{0.8cm}

~\vfill{}

\section*{Abstrakt}
\vskip 0.5em

\sloppy

Tato práce zkoumá aplikaci embeddingů slov a vět v modelu \ac{RAG} pro úlohy \ac{QA} zaměřené na fakta s využitím technických příruček.
Studie se zabývá účinností tradičních embeddingů FastText a pokročilých modelů založených na transformátoru, jako je \ac{BERT}, při zachycování sémantických vztahů v textu.
Kvalitu těchto reprezentací hodnotíme pomocí analogových testů a analýzy confusion matrix na korpusu UPV.
Následně vybereme optimální reprezentace pro algoritmy \ac{RAG} a posoudíme jejich vliv na faktickou přesnost a výpočetní efektivitu během \ac{QA}.
Analýzou výkonu s různými velikostmi textových fragmentů se snažíme identifikovat optimální konfiguraci pro faktické \ac{RAG} v technických oborech.
Výzkum přispívá k oblasti zpracování přirozeného jazyka (\ac{NLP}) tím, že poskytuje poznatky o výběru efektivních reprezentací, které vyvažují faktickou přesnost a výpočetní efektivitu pro systémy \ac{QA}.


\vskip 1em

{\bf Klíčová slova} \KlicovaSlova

\vskip 2.5cm

\end{changemargin}
