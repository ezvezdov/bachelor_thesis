%!TEX root = ../main.tex

\begin{changemargin}{0.8cm}{0.8cm}

~\vfill{}

\section*{Abstrakt}
\vskip 0.5em

\sloppy

Tato práce zkoumá vývoj metod reprezentace textu, od tradičních technik jako FastText až po sofistikované modely založené na transformátorech, jako je \ac{BERT}.
Studie hodnotí tyto reprezentace prostřednictvím testů analogie a analýzy matic záměn, přičemž využívá korpus UPV pro komplexní posouzení.

V pozdější části výzkumu se pozornost přesouvá k optimalizaci reprezentací textu pro algoritmy \ac{RAG}.
Výzkum si klade za cíl identifikovat nejúčinnější vektory a určit optimální velikost textových bloků pro úkoly \ac{QA}, zejména v oblasti generování odpovědí v přirozeném jazyce z technických manuálů.
Provádí se důkladné hodnocení s cílem doporučit optimální model reprezentace, který vyvažuje faktickou přesnost a výpočetní efektivitu.

\vskip 1em

{\bf Klíčová slova} \KlicovaSlova

\vskip 2.5cm

\end{changemargin}
